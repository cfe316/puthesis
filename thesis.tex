% ****************************************************************************************** % Dissertation template and document class for Princeton University
% Author  : Jeffrey Scott Dwoskin <jdwoskin@princeton.edu>
% Adapted from: http://www.math.princeton.edu/graduate/tex/puthesis.html
% ****************************************************************************************** %


%%% For print copies
%% set 'singlespace' option to set entire thesis to single space, and define "\printmode" to remove all hyperlinks for printed copies of the thesis. Delete all output files before changing this mode -- it will turn hyperref package on and off
%\documentclass[12pt,lot, lof, singlespace]{puthesis}
% \newcommand{\printmode}{}

%%% For the electronic copy, use doublespacing, define "\proquestmode" to use outlined links, instead of colored links.
\documentclass[12pt,lot,lof]{puthesis}
% lot : make the List of Tables
% lof : make the List of Figures

% \newcommand{\proquestmode}{}
% I prefer proquestmode to be off for electronic copies for normal use, since the colored links are less distracting. However when printed in black and white, the colored links are difficult to read.

%%% For early drafts without the list of tables and list of figures
% Also see the "ifodd" command below to disable more frontmatter
%\documentclass[12pt]{puthesis}

%%%%%%%%%%%%%%%%%%%%%%%%%%%%%%%%%%%%%%%%%%%%%%%%%%%%%%%%%%%%%\
%%%% Author & title page info

\title{Dissertation Template for Princeton University}

\submitted{June 2010}  % degree conferral date (January, April, June, September, or November)
\copyrightyear{2010}  % year in which the copyright is secured by publication of the dissertation.
\author{First Middle Last}
\advisor{Professor Smith}  %replace with the full name of your advisor
%\departmentprefix{Program in}  % defaults to "Department of", but programs need to change this.
\department{Electrical Engineering}

%%%%%%%%%%%%%%%%%%%%%%%%%%%%%%%%%%%%%%%%%%%%%%%%%%%%%%%%%%%%%\
%%%% Tweak float placements
% From: http://mintaka.sdsu.edu/GF/bibliog/latex/floats.html "Controlling LaTeX Floats"
% and based on: http://www.tex.ac.uk/cgi-bin/texfaq2html?label=floats
% LaTeX defaults listed at: http://people.cs.uu.nl/piet/floats/node1.html

% Alter some LaTeX defaults for better treatment of figures:
    % See p.105 of "TeX Unbound" for suggested values.
    % See pp. 199-200 of Lamport's "LaTeX" book for details.
    %   General parameters, for ALL pages:
    \renewcommand{\topfraction}{0.85} % max fraction of floats at top
    \renewcommand{\bottomfraction}{0.6} % max fraction of floats at bottom
    %   Parameters for TEXT pages (not float pages):
    \setcounter{topnumber}{2}
    \setcounter{bottomnumber}{2}
    \setcounter{totalnumber}{4}     % 2 may work better
    \setcounter{dbltopnumber}{2}    % for 2-column pages
    \renewcommand{\dbltopfraction}{0.66}  % fit big float above 2-col. text
    \renewcommand{\textfraction}{0.15}  % allow minimal text w. figs
    %   Parameters for FLOAT pages (not text pages):
    \renewcommand{\floatpagefraction}{0.66} % require fuller float pages
    % N.B.: floatpagefraction MUST be less than topfraction !!
    \renewcommand{\dblfloatpagefraction}{0.66}  % require fuller float pages

% The documentclass already sets parameters to make a high penalty for widows and orphans.

%%%%%%%%%%%%%%%%%%%%%%%%%%%%%%%%%%%%%%%%%%%%%%%%%%%%%%%%%%%%%\
%%%% Use packages
%%%% I've found the best quick reference for packages is the 'texdoc' command.
%%%% It usually opens up a pdf of the documentation.

% Improves the output font encoding: see
% https://tex.stackexchange.com/questions/664/why-should-i-use-usepackaget1fontenc
\usepackage[T1]{fontenc}

%\usepackage{amsfonts}

%%% For figures
\usepackage{graphicx}
%\usepackage{subfig,rotate}

%%% for comments
\usepackage{verbatim}

%%% For tables
\usepackage{multirow}
% Longtable lets you have tables that span multiple pages.
\usepackage{longtable}

% Booktabs produces far nicer tables than the standard LaTeX tables.
%   see: http://en.wikibooks.org/wiki/LaTeX/Tables
\usepackage{booktabs}

%set parameters for longtable:
% default caption width is 4in for longtable, but wider for normal tables
\setlength{\LTcapwidth}{\textwidth}

%controlling hyphenations
\usepackage[english]{babel}
\hyphenation{Mathematica MATLAB} % do not hyphenate
\hyphenation{to-ka-mak} % where to hyphenate

%For printing numbers and units. Highly recommend!!
\usepackage{siunitx}

%%%% Additional packages you may not have considered
% For chemicals and chemical reactions. Also highly recommend!!
% \usepackage[version=4]{mhchem}

% For sideways tables or figure like for a full-page extra-wide table
% PDF readers will automatically rotate just the one page.
% \usepackage{pdflscape}
% \usepackage{afterpage}

% For the rest of your thesis
% \usepackage{lipsum}

% Versatile package for making more professional figures and tables.
% For example, figures with captions on the side.
% Highly recommended!See `texdoc floatrow`
% \usepackage[capbesideposition=outside,
%             facing=yes, footnoterule=none, footskip=.35\skip\footins,
%             capbesidesep=quad]{floatrow}

% for fancy fractions: \sfrac command.
% \usepackage{xfrac}

% For more professional tables: allows aligning columns on the decimal point
% \usepackage{dcolumn}
% \newcolumntype{d}[1]{D{.}{\cdot}{#1}}

% For caption on another page when you have a very big figure
% \usepackage{ccaption}
% \usepackage[style=base,tableposition=top]{caption}

% for subfigures (two figures side by side)
% \usepackage{subcaption}

%% The 'layout' package allows you to make sure the page layout
%% (margins, body width, etc) are correct.
%% To use it, uncomment this and the 'layout' line just after \begin{document}
%% Just remember one tex point is 1/72.27 inches.
\usepackage{layout}

%%%% Improved bibliography
%%%% Most of this is from stackoverflow.
\usepackage[maxbibnames=3,
            bibstyle=numeric-comp,
            citestyle=numeric-comp,
            eprint=false,
            giveninits=true,
            backref=true, % backreferences are totally cool because you can see how many times and where a source was cited.
            backend=biber]{biblatex}

\AtEveryBibitem{\clearfield{issn}\clearfield{pagetotal}}
\AtEveryCitekey{\clearfield{issn}}

% Allow proper formatting of 'letter's which I think was how I cited emails
% or other documents
\DeclareBibliographyAlias{letter}{misc}

% goes along with the 'backreferences' option
% replace the default backreference string "(cit. on p. 4)" with "(p. 4)"
\DefineBibliographyStrings{english}{%
     backrefpage  = {p.}, % for single page number
     backrefpages = {pp.} % for multiple page numbers
}

% IIRC this makes it so that you print the DOI (as a hyperlink to doi.org) rather than a hyperlink to the journal article
% but it will still print a standard URL if no DOI is available. I did this because having two hyperlinks is too much.
\renewbibmacro*{doi+eprint+url}{
  \iftoggle{bbx:url}
    {\iffieldundef{doi}{\usebibmacro{url+urldate}}{}}
    {}
   \newunit\newblock
   \iftoggle{bbx:eprint}{\usebibmacro{eprint}}
   {}
   \newunit\newblock
   \iftoggle{bbx:doi}
   {\printfield{doi}}
   {}}

\renewbibmacro*{volume+number+eid}{%
  \textbf{\printfield{volume}}%
%  \setunit*{\adddot}% DELETED
  \setunit*{\addnbthinspace}% NEW (optional); there's also \addnbthinspace
  \printfield{number}%
  \setunit{\addcomma\space}%
  \printfield{eid}}
\DeclareFieldFormat[article]{number}{\mkbibparens{#1}}

\renewbibmacro{in:}{\ifentrytype{article}{}{\printtext{\bibstring{in}\intitlepunct}}}

\renewbibmacro*{issue+date}{%
  \printtext[]{\addcomma\space%
    %\printfield{issue}%      % these are only important for a journal with issues like "Spring".
%    \setunit*{\addspace}%
    \usebibmacro{date}}%
  \newunit}

\addbibresource{thesis.bib}

\usepackage[dvipsnames, svgnames]{xcolor}
% Custom colors of hyperref links:
\colorlet{myIntColor}{DarkRed} % for links to equations, figures, and sections
\colorlet{myCiteColor}{DarkGreen} % citations
\colorlet{myExtColor}{MidnightBlue} % external hyperlinks / urls
%%%%%%%%%%%%%%%%%%%%%%%%%%%%%%%%%%%%%%%%%%%%%%%%%%%%%%%%%%
%%% Printed vs. online formatting
\ifdefined\printmode

% Printed copy
% While obviously the printed copy doesn't need hyperlinks,
% This allows you to keep any \autoref or \cref commands
\usepackage[hyperfootnotes=false, hidelinks]{hyperref}

\else

\ifdefined\proquestmode
% ProQuest copy -- https://library.princeton.edu/special-collections/policies/masters-theses-and-phd-dissertations-submission-guidelines

% ProQuest requires a double spaced version (set previously). They will take an electronic copy, so we want links in the pdf, but also copies may be printed or made into microfilm in black and white.
% One option is to have 'outlined' links but I think that is super ugly, so I just make the proquest links black. This effectively means printmode is the same as proquestmode (except for single-spacing it)
% This also means that only I have the 'definitive' copy of the thesis (with the nicely colored links) to distribute as I like. This helps dissipate some of my rage against Proquest. -JAS
\usepackage[hyperfootnotes=false, hidelinks]{hyperref}

\else
% Online copy
% adds internal linked references, pdf bookmarks, etc

% turn all references and citations into hyperlinks:
%  -- not for printed copies
%  -- automatically includes url package
% options:
%   colorlinks makes links by coloring the text instead of putting a rectangle around the text.
%   linktoc=all makes both the section
\usepackage[hyperfootnotes=false]{hyperref}
\hypersetup{colorlinks,
linkcolor=myIntColor,
citecolor=myCiteColor,
urlcolor =myExtColor,
menucolor=myIntColor,
}

\fi % proquest or online formatting

% copy the already-set title and author to use in the pdf properties
\makeatletter
\hypersetup{pdftitle=\@title,pdfauthor=\@author}
\hypersetup{bookmarksnumbered, linktoc=all}
\makeatother

\fi % printed or online formatting

% The cleveref packages is set AFTER hyperref. Highly recommended!
% Allows writing \cref{eq:pythagoras} instead of Eq.~(\ref{eq:pythagoras})
% and more customization of which parts of the reference are links
\usepackage{cleveref}

%%%%%%%%%%%%%%%%%%%%%%%%%%%%%%%%%%%%%%%%%%%%%%%%%%%%%%%%%%%%%

%%%% Proper footnote sizing: PU says "Font size should be equivalent in scale to 10 point Arial or 12 point Times New Roman. These rules apply to captions, and bibliographies. Footnotes and endnotes can be one point smaller than the body of the text."
\makeatletter
\renewcommand\footnotesize{%
   \@setfontsize\footnotesize\@xipt\@xiipt
   \abovedisplayskip 10\p@ \@plus2\p@ \@minus5\p@
   \abovedisplayshortskip \z@ \@plus3\p@
   \belowdisplayshortskip 6\p@ \@plus3\p@ \@minus3\p@
   \def\@listi{\leftmargin\leftmargini
               \topsep 6\p@ \@plus2\p@ \@minus2\p@
               \parsep 3\p@ \@plus2\p@ \@minus\p@
               \itemsep \parsep}%
   \belowdisplayskip \abovedisplayskip
}
\makeatother

%%%%%%%%%%%%%%%%%%%%%%%%%%%%%%%%%%%%%%%%%%%%%%%%%%%%%%%%%%%%%\
%%%% Define commands

% Define any custom commands that you want to use.
% For example, highlight notes for future edits to the thesis
%\newcommand{\todo}[1]{\textbf{\emph{TODO:}#1}}


% create an environment that will indent text
% see: http://latex.computersci.org/Reference/ListEnvironments
%   \raggedright makes them left aligned instead of justified
\newenvironment{indenttext}{
    \begin{list}{}{ \itemsep 0in \itemindent 0in
    \labelsep 0in \labelwidth 0in
    \listparindent 0in
    \topsep 0in \partopsep 0in \parskip 0in \parsep 0in
    \leftmargin 1em \rightmargin 0in
    \raggedright
    }
    \item
  }
  {\end{list}}

% another environment that's an indented list, with no spaces between items -- if we want multiple items/lines. Useful in tables. Use \item inside the environment.
%   \raggedright makes them left aligned instead of justified
\newenvironment{indentlist}{
    \begin{list}{}{ \itemsep 0in \itemindent 0in
    \labelsep 0in \labelwidth 0in
    \listparindent 0in
    \topsep 0in \partopsep 0in \parskip 0in \parsep 0in
    \leftmargin 1em \rightmargin 0in
    \raggedright
    }

  }
  {\end{list}}



%%%%%%%%%%%%%%%%%%%%%%%%%%%%%%%%%%%%%%%%%%%%%%%%%%%%%%%%%%%%%\
%%%% Front-matter

% For early drafts, you may want to disable some of the frontmatter. Simply change this to "\ifodd 1" to do so.
\ifodd 0
% front-matter disabled while writing chapters
\renewcommand{\maketitlepage}{}
\renewcommand*{\makecopyrightpage}{}
\renewcommand*{\makeabstract}{}

% you can just skip the \acknowledgements and \dedication commands to leave out these sections.

\else


\abstract{
% Abstract can be any length, but should be max 350 words for a Dissertation for ProQuest's print indicies (150 words for a Master's Thesis) or it will be truncated for those uses.
\input{abstract}
}

\acknowledgements{
%I would like to thank...
\input{acknowledgements}
}

\dedication{To my parents.}

\fi  % disable frontmatter


%%%%%%%%%%%%%%%%%%%%%%%%%%%%%%%%%%%%%%%%%%%%%%%%%%%%%%%%%%%%%\
%%%% Hide some chapters

%%% If you want to produce a pdf that includes only certain chapters, specify them with includeonly, in addition to including all chapters below.
%\includeonly{ch-intro/chapter-intro}
%%% You can also specify multiple chapters.
%\includeonly{ch-intro/chapter-intro,ch-usage/chapter-usage}
%\includeonly{chap1,chap2,chap3}


%%%%%%%%%%%%%%%%%%%%%%%%%%%%%%%%%%%%%%%%%%%%%%%%%%%%%%%%%%%%%
%%%% Notes:

% Footnotes should be placed after punctuation.\footnote{place here.}
% Generally, place citations before the period~\cite{anotherauthor}.
% The proper usage for i.e., and e.g., include commas ``(e.g., option A, option B)''

%%%%%%%%%%%%%%%%%%%%%%%%%%%%%%%%%%%%%%%%%%%%%%%%%%%%%%%%%%%%%
%%%% Import chapters

\begin{document}

% Activate to verify the layout
% \layout

{\hypersetup{linkcolor=black}
\makefrontmatter
}
% need to re-invoke this command from puthesis.cls, in order to have double spacing in the rest of the document.
% It's invoked inside \makefrontmatter but then goes away since it's inside braces.
\bodyspacing

% If you've disabled frontmatter, you can insert the toc manually
%\tableofcontents\clearpage

% \include lets us split up the document (and each include starts a new page):
\include{ch-intro/chapter-intro}
\include{ch-pastwork/chapter-pastwork}
\include{ch-usage/chapter-usage}
\include{ch-conclusion/chapter-conclusion}
\appendix % all chapters following will be labeled as appendices
\include{ch-appendicies/implementation}
\include{ch-appendicies/printing}


% Make the bibliography single spaced
\singlespacing

% add the Bibliography to the Table of Contents
\cleardoublepage
\ifdefined\phantomsection
  \phantomsection  % makes hyperref recognize this section properly for pdf link
\else
\fi
\addcontentsline{toc}{chapter}{Bibliography}

\printbibliography

\end{document}
